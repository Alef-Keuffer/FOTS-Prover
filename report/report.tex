
\documentclass[11pt,a4paper]{report}%especifica o tipo de documento que tenciona escrever: carta, artigo, relatório... neste caso é um relatório
% [11pt,a4paper] Define o tamanho principal das letras do documento. caso não especifique uma delas, é assumido 10pt
% a4paper -- Define o tamanho do papel.

\usepackage[portuges]{babel}%Babel -- irá activar automaticamente as regras apropriadas de hifenização para a língua todo o
                                   %-- o texto gerado é automaticamente traduzido para Português.
                                   %  Por exemplo, “chapter” irá passar a “capítulo”, “table of contents” a “conteúdo”.
                                   % portuges -- específica para o Português.
\usepackage[utf8]{inputenc} % define o encoding usado texto fonte (input)--usual "utf8" ou "latin1

\usepackage{graphicx} %permite incluir graficos, tabelas, figuras
\usepackage{url} % para utilizar o comando \url{}
\usepackage{enumerate} %permite escolher, nas listas enumeradas, se os iems sao marcados com letras ou numeros-romanos em vez de numeracao normal

%\usepackage{apalike} % gerar biliografia no estilo 'named' (apalike)

\usepackage{color} % Para escrever em cores
\usepackage{xcolor}

\usepackage{multirow} %tabelas com multilinhas
\usepackage{array} %formatação especial de tabelas em array

\usepackage[pdftex]{hyperref} % transformar as referências internas do seu documento em hiper-ligações.

% Para autómatos
\usepackage{tikz}
\usetikzlibrary{automata,arrows,positioning, arrows.meta, shapes.geometric}

\usepackage{amsmath,amssymb,amsfonts}
\usepackage{pdfpages}
\usepackage{float} % Image location specifier

%Exemplos de fontes -- nao e vulgar mudar o tipo de fonte
%\usepackage{tgbonum} % Fonte de letra: TEX Gyre Bonum
%\usepackage{lmodern} % Fonte de letra: Latin Modern Sans Serif
%\usepackage{helvet}  % Fonte de letra: Helvetica
%\usepackage{charter} % Fonte de letra:Charter

\definecolor{saddlebrown}{rgb}{0.55, 0.27, 0.07} % para definir uma nova cor, neste caso 'saddlebrown'

\usepackage{listings}  % para utilizar blocos de texto verbatim no estilo 'listings'
%paramerização mais vulgar dos blocos LISTING - GENERAL

%
%\lstset{ %
%	language=Java,							% choose the language of the code
%	basicstyle=\ttfamily\footnotesize,		% the size of the fonts that are used for the code
%	keywordstyle=\bfseries,					% set the keyword style
%	%numbers=left,							% where to put the line-numbers
%	numberstyle=\scriptsize,				% the size of the fonts that are used for the line-numbers
%	stepnumber=2,							% the step between two line-numbers. If it's 1 each line
%											% will be numbered
%	numbersep=5pt,							% how far the line-numbers are from the code
%	backgroundcolor=\color{white},			% choose the background color. You must add \usepackage{color}
%	showspaces=false,						% show spaces adding particular underscores
%	showstringspaces=false,					% underline spaces within strings
%	showtabs=false,							% show tabs within strings adding particular underscores
%	frame=none,								% adds a frame around the code
%	%abovecaptionskip=-.8em,
%	%belowcaptionskip=.7em,
%	tabsize=2,								% sets default tabsize to 2 spaces
%	captionpos=b,							% sets the caption-position to bottom
%	breaklines=true,						% sets automatic line breaking
%	breakatwhitespace=false,				% sets if automatic breaks should only happen at whitespace
%	title=\lstname,							% show the filename of files included with \lstinputlisting;
%											% also try caption instead of title
%	escapeinside={\%*}{*)},					% if you want to add a comment within your code
%	morekeywords={*,...}					% if you want to add more keywords to the set
%}

\usepackage{xspace} % deteta se a seguir a palavra tem uma palavra ou um sinal de pontuaçao se tiver uma palavra da espaço, se for um sinal de pontuaçao nao da espaço

\parindent=0pt %espaço a deixar para fazer a  indentação da primeira linha após um parágrafo
\parskip=2pt % espaço entre o parágrafo e o texto anterior

\setlength{\oddsidemargin}{-1cm} %espaço entre o texto e a margem
\setlength{\textwidth}{18cm} %Comprimento do texto na pagina
\setlength{\headsep}{-1cm} %espaço entre o texto e o cabeçalho
\setlength{\textheight}{23cm} %altura do texto na pagina

% comando '\def' usado para definir abreviatura (macros)
% o primeiro argumento é o nome do novo comando e o segundo entre chavetas é o texto original, ou sequência de controle, para que expande
\def\proj{\emph{Projeto}\xspace}
\def\pdr{Property Directed Reachability\xspace}
\def\bmc{Bounded Model-Checking\xspace}
\def\imc{Interpolant-Based Model-Checking\xspace}
\def\kind{``\textit{$k$-induction}''\xspace}
\def\titulo#1{\section{#1}}    %no corpo do documento usa-se na forma '\titulo{MEU TITULO}'
\def\area#1{{\em \'{A}rea: #1}\\[0.2cm]}
\def\super#1{{\em Supervisor: #1}\\ }
\def\resumo{\underline{Resumo}:\\ }

%\input{LPgeneralDefintions} %permite ler de um ficheiro de texto externo mais definições

\title{UC Projeto\\
      3º ano Licenciatura em Ciências da Computação \\
      Construção de um ferramenta genérica de verificação SAT para propriedades de segurança
e animação de sistemas de transição de 1º ordem (FOTS)
      } %Titulo do documento
%\title{Um Exemplo de Artigo em \LaTeX}
\author{Alef Keuffer\\ (A91683) \and Alexandre Baldé\\ (A70373)
         \and Bruno Machado\\ (A91680) \and Pedro Pereira\\ (A88062) \\ \\
        Supervisor: Professor José Manuel Esgalhado Valença
       } %autores do documento
\date{\today} %data

\begin{document} % corpo do documento
\maketitle % apresentar titulo, autor e data

\begin{abstract}  % resumo do documento
Neste relatório explicar-se-á ...
\end{abstract}

\tableofcontents % Insere a tabela de indice
\listoffigures % Insere a tabela de indice figuras
%\listoftables % Insere a tabela de indice tabelas

\chapter{Introdução} \label{chap:intro} %referência cruzada

Este relatório contém a descrição do projeto realizado pelos autores para a
UC de Projeto da Licenciatura em Ciências da Computação, para o ano letivo de 2021/2022.

\section{Estrutura do Relatório}

A estrutura do relatório é a seguinte:
\begin{itemize}
\item No capítulo~\ref{chap:state_of_the_art} faz-se uma análise do trabalho
  já existente na área, e das referência usadas para o projeto.

\item No capítulo~\ref{chap:analysis} explicam-se alguns aspetos mais técnicos e concretos da implementação.

\item No capítulo~\ref{chap:case_study} apresenta-se um case de estudo com um FOTS que servirá para
apresentação das funcionalidades desenvolvidas.

\item No capítulo~\ref{chap:concl} termina-se o relatório com as conclusões e o trabalho futuro.
\end{itemize}

\section{Problema em análise}

\section{Resolução e Estratégias adotadas}

\section{Agradecimentos}

\chapter{Estado de arte} \label{chap:state_of_the_art} %referência cruzada

\chapter{Análise do trabalho} \label{chap:analysis}

\newpage
\section{\kind e \bmc}

\newpage
\section{\imc}

\newpage
\section{\pdr}

\chapter{Caso de Estudo}\label{chap:case_study}

\chapter{Conclusão} \label{chap:concl}

Conclui-se desta forma a apresentação do trabalho desenvolvido pelos autores
para a UC \proj no ano letivo 2021/2022.

\section{Comentários}

\section{Trabalho Futuro}

\appendix % apendice
\chapter{Excertos de Código Utilizado no Projeto}

\chapter{Referência para repositório \textit{GitHub} com código fonte}

\newpage

%-- Fim do documento -- inserção das referencias bibliográficas

%\bibliographystyle{plain} % [1] Numérico pela ordem de citação ou ordem alfabetica
\bibliographystyle{alpha} % [Hen18] abreviação do apelido e data da publicação
%\bibliographystyle{apalike} % (Araujo, 2018) apelido e data da publicação
                             % --para usar este estilo descomente no inicio o comando \usepackage{apalike}

\bibliography{bibLayout} %input do ficheiro de referencias bibliograficas

\end{document}